\documentclass[11pt]{article}
\usepackage{amsmath,amssymb,geometry}
\geometry{margin=1in}

\title{Unified Rigidity Coin (URC): Protocol Contract and Security Model}
\author{Inacio F. Vasquez}
\date{\today}

\begin{document}
\maketitle

\section*{Protocol Contract (Normative)}

URC is defined by the following invariants.

\paragraph{(I1) Cryptographic ownership.}
A spend of UTXO $u$ is valid only if the input public key equals the stored UTXO public key.

\paragraph{(I2) Ed25519 authentication.}
For every non-coinbase transaction $tx$ with input $(pk, sig)$,
\[
\mathrm{Verify}\big(pk,\ \mathrm{canon}(tx \setminus sig),\ sig\big)=\top.
\]

\paragraph{(I3) No inflation.}
For every non-coinbase transaction $tx$,
\[
\sum \mathrm{in}(tx) \ge \sum \mathrm{out}(tx).
\]

\paragraph{(I4) No double-spend.}
Every UTXO key may appear in inputs at most once (block-local and mempool-global).

\paragraph{(I5) Checkpoint safety.}
If $h\in \mathrm{dom}(\mathrm{CHECKPOINTS})$, then
\[
\mathrm{hash}(B_h)=\mathrm{CHECKPOINTS}[h].
\]

\paragraph{(I6) Proof-of-Work validity.}
\[
\mathrm{int}(\mathrm{SHA256}(\mathrm{canon}(\mathrm{header})),16) < T_0.
\]

\paragraph{(I7) Fee-based mempool ordering.}
Transactions are ordered by fee-per-byte:
\[
tx_1 \prec tx_2 \iff 
\frac{\mathrm{fee}(tx_1)}{|\mathrm{canon}(tx_1)|} >
\frac{\mathrm{fee}(tx_2)}{|\mathrm{canon}(tx_2)|}.
\]

\section*{Threat Model and Non-Claim}

The adversary may control:
\begin{itemize}
\item the network,
\item message ordering,
\item and arbitrary computational resources.
\end{itemize}

URC explicitly does \textbf{not} claim asymptotic security accumulation.  
Security is structural, not economic.

\end{document}

